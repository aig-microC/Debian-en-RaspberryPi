%% Generated by Sphinx.
\def\sphinxdocclass{report}
\documentclass[letterpaper,10pt,spanish]{sphinxmanual}
\ifdefined\pdfpxdimen
   \let\sphinxpxdimen\pdfpxdimen\else\newdimen\sphinxpxdimen
\fi \sphinxpxdimen=.75bp\relax
\ifdefined\pdfimageresolution
    \pdfimageresolution= \numexpr \dimexpr1in\relax/\sphinxpxdimen\relax
\fi
%% let collapsible pdf bookmarks panel have high depth per default
\PassOptionsToPackage{bookmarksdepth=5}{hyperref}


\PassOptionsToPackage{warn}{textcomp}
\usepackage[utf8]{inputenc}
\ifdefined\DeclareUnicodeCharacter
% support both utf8 and utf8x syntaxes
  \ifdefined\DeclareUnicodeCharacterAsOptional
    \def\sphinxDUC#1{\DeclareUnicodeCharacter{"#1}}
  \else
    \let\sphinxDUC\DeclareUnicodeCharacter
  \fi
  \sphinxDUC{00A0}{\nobreakspace}
  \sphinxDUC{2500}{\sphinxunichar{2500}}
  \sphinxDUC{2502}{\sphinxunichar{2502}}
  \sphinxDUC{2514}{\sphinxunichar{2514}}
  \sphinxDUC{251C}{\sphinxunichar{251C}}
  \sphinxDUC{2572}{\textbackslash}
\fi
\usepackage{cmap}
\usepackage[T1]{fontenc}
\usepackage{amsmath,amssymb,amstext}
\usepackage{babel}



\usepackage{tgtermes}
\usepackage{tgheros}
\renewcommand{\ttdefault}{txtt}



\usepackage[Sonny]{fncychap}
\ChNameVar{\Large\normalfont\sffamily}
\ChTitleVar{\Large\normalfont\sffamily}
\usepackage{sphinx}

\fvset{fontsize=auto}
\usepackage{geometry}


% Include hyperref last.
\usepackage{hyperref}
% Fix anchor placement for figures with captions.
\usepackage{hypcap}% it must be loaded after hyperref.
% Set up styles of URL: it should be placed after hyperref.
\urlstyle{same}

\addto\captionsspanish{\renewcommand{\contentsname}{Contenido:}}

\usepackage{sphinxmessages}
\setcounter{tocdepth}{1}



\title{Debian nativo en una RPi}
\date{16 de septiembre de 2023}
\release{2.0}
\author{Angel de la Iglesia}
\newcommand{\sphinxlogo}{\vbox{}}
\renewcommand{\releasename}{Versión}
\makeindex
\begin{document}

\ifdefined\shorthandoff
  \ifnum\catcode`\=\string=\active\shorthandoff{=}\fi
  \ifnum\catcode`\"=\active\shorthandoff{"}\fi
\fi

\pagestyle{empty}
\sphinxmaketitle
\pagestyle{plain}
\sphinxtableofcontents
\pagestyle{normal}
\phantomsection\label{\detokenize{index::doc}}



\chapter{Introducción}
\label{\detokenize{index:introduccion}}
\sphinxAtStartPar
En este documento se describe como instalar el \sphinxstyleemphasis{Sistema Operativo Debian} en una \sphinxhref{https://www.raspberrypi.com/}{Raspberry Pi}. Esta tarjeta, que es un ordenador completo, tiene su propio sistema operativo oficial: \sphinxstyleemphasis{Raspberry Pi OS (de 32 o 64\sphinxhyphen{}bits)} que se puede obtener de \sphinxurl{https://www.raspberrypi.com/software/operating-systems/} y que es un derivado de \sphinxstyleemphasis{Debian} \sphinxstyleemphasis{estable}, optimizado para esta tarjeta. Yo prefiero trabajar con un sistema \sphinxstyleemphasis{Debian} nativo y en su versión \sphinxstyleemphasis{testing}, si ya la versión estable tiene más de 3 o 6 meses de antigüedad.

\sphinxAtStartPar
A continuación te muestro el procedimiento para realizar una instalación de \sphinxstyleemphasis{Debian} en una tarjeta \sphinxstyleemphasis{microSD} ya sea para la versión \sphinxstyleemphasis{estable} o la \sphinxstyleemphasis{testing}.

\sphinxAtStartPar
Básicamente, la selección de la imagen de partida (ver \sphinxurl{https://wiki.debian.org/RaspberryPi} y la respuesta 6 de \sphinxurl{https://raspi.debian.net/faq/}) depende de la \sphinxstyleemphasis{familia} a la que pertenece la tarjeta de la \sphinxstyleemphasis{Raspberry Pi}.
\begin{itemize}
\item {} 
\sphinxAtStartPar
Las \sphinxstyleemphasis{Raspberry Pi Zero}, \sphinxstyleemphasis{Raspberry Pi Zero W} o \sphinxstyleemphasis{Raspberry Pi Zero WH} utilizan un chip que se corresponde con una arquitectura de 32 bits de tipo \sphinxstyleemphasis{armel} lo que supone que no dispone de una unidad de coma (punto en inglés) flotante (\sphinxstyleemphasis{FPU} \sphinxstyleemphasis{Floating Point Unit}) para las operaciones de cálculo. Los cálculos los hace mediante software y por lo tanto es más lento que si dispusiera de una \sphinxstyleemphasis{FPU}. En la notación que utiliza \sphinxstyleemphasis{Debian} es de la \sphinxstylestrong{familia 0/1}. No consideraré la \sphinxstyleemphasis{Raspberry Pi Zero} porque no tiene conectividad con la que acceder a \sphinxstyleemphasis{Internet} mientras que las que si la tienen, \sphinxstyleemphasis{W} y \sphinxstyleemphasis{WH}, solo se diferencian en si tiene (\sphinxstyleemphasis{WH}) o no (\sphinxstyleemphasis{W}) la tira de de pines para conexiones (\sphinxstyleemphasis{GPIO}) soldada. A esta familia pertenecen también las \sphinxstyleemphasis{Raspberry Pi A, B, A+, B+ y las mencionadas Zero, Zero W})

\item {} 
\sphinxAtStartPar
La \sphinxstyleemphasis{Raspberry Pi 2} tiene arquitectura \sphinxstyleemphasis{amrhf} de 32 bits con 1 GB de memoria \sphinxstyleemphasis{RAM}. En la notación que utiliza \sphinxstyleemphasis{Debian} es de la \sphinxstylestrong{familia 2}

\item {} 
\sphinxAtStartPar
La \sphinxstyleemphasis{Raspberry Pi Zero 2W} es de 64 bits con \sphinxstyleemphasis{FPU}. Su arquitectura es \sphinxstyleemphasis{armhf}. Tiene, como la \sphinxstyleemphasis{Zero W} 512 MB de memoria \sphinxstyleemphasis{RAM} por lo que si instalas un escritorio (sí, se puede instalar un escritorio y tener un entorno gráfico) no será muy rápido (pero un poco más rápida que si lo pruebas en una \sphinxstyleemphasis{Zero W}). En la notación que utiliza \sphinxstyleemphasis{Debian} es de la \sphinxstylestrong{familia 3}. A esta familia pertenecen también las \sphinxstyleemphasis{Raspberry Pi 3, 3A+, 3B+ y la mencionada Zero 2 W}).

\item {} 
\sphinxAtStartPar
Las \sphinxstyleemphasis{Raspberry Pi 4} y \sphinxstyleemphasis{Raspberry Pi 400} son las versiones más potentes, con arquitectura \sphinxstyleemphasis{arm64} de 64 bits y 4 GB de memoria. La \sphinxstyleemphasis{Raspberry Pi 4} dispone de una versión con 8 GB de memoria \sphinxstyleemphasis{RAM}. En la notación que utiliza \sphinxstyleemphasis{Debian} es de la \sphinxstylestrong{familia 4}.

\end{itemize}

\sphinxAtStartPar
Este documento está bajo  \sphinxhref{https://creativecommons.org/licenses/by-sa/4.0/}{Licencia Creative Commons Atribución\sphinxhyphen{}CompartirIgual 4.0 Internacional}. Todas las marcas comerciales referidas en este documento pertenecen a sus legítimos propietarios.

\begin{figure}[htbp]
\centering

\noindent\sphinxincludegraphics[scale=0.25]{{CC}.png}
\end{figure}

\sphinxAtStartPar
La última versión de este documento está en \sphinxurl{https://github.com/aig-microC/Debian-en-RaspberryPi}.

\newpage

\sphinxstepscope


\section{Procedimiento para instalar \sphinxstyleemphasis{Debian} «nativo» en una \sphinxstyleemphasis{Raspberry Pi}}
\label{\detokenize{DebianNativoRPi:procedimiento-para-instalar-debian-nativo-en-una-raspberry-pi}}\label{\detokenize{DebianNativoRPi::doc}}
\sphinxAtStartPar
Lo primero que hay que hacer es descargar la imagen básica desde el enlace \sphinxurl{https://raspi.debian.net/}. Hay dos posibilidades: descargar la imagen creada diariamente (\sphinxurl{https://raspi.debian.net/daily-images/}) o de la versión probadas(\sphinxurl{https://raspi.debian.net/tested-images/}). Yo suelo utilizar las imágenes creadas diariamente y si encuentro problemas en la instalación me descargo la imagen testeada.

\sphinxAtStartPar
Una vez descargada hay que escribirla en una tarjeta \sphinxstyleemphasis{microSD}. Para ello utilizaremos, si tenemos instalado el \sphinxstyleemphasis{Raspberry Pi OS} en una \sphinxstyleemphasis{RPi 4 o 400} el programa \sphinxhref{https://www.raspberrypi.com/software/}{rpi\sphinxhyphen{}imager} que que está disponible en este sistema operativo o si estamos en un sistema \sphinxstyleemphasis{Debian} se puede compilar e instalar siguiendo las instrucciones que se pueden ver en \sphinxurl{https://github.com/raspberrypi/rpi-imager}.

\begin{figure}[htbp]
\centering
\capstart

\noindent\sphinxincludegraphics[scale=0.8]{{rpi-imager}.png}
\caption{rpi\sphinxhyphen{}imager. Programa para copiar una imagen del SO}\label{\detokenize{DebianNativoRPi:id1}}\end{figure}

\sphinxAtStartPar
Seleccionamos \sphinxstyleemphasis{CHOSE OS} y vamos al final para seleccionar «\sphinxstylestrong{Use custom}, \sphinxstyleemphasis{Select a custom .img from your computer}» y buscamos la imagen que acabamos de descargar. A continuación seleccionamos «\sphinxstyleemphasis{CHOSE STORAGE}» y seleccionamos la memoria \sphinxstyleemphasis{microSD}. Y por último pulsamos «\sphinxstyleemphasis{WRITE}» para escribir la imagen en la tarjeta y terminar.

\sphinxAtStartPar
También lo podemos hacer utilizando la línea de comandos, con el procedimiento que se describe en \sphinxurl{https://raspi.debian.net/how-to-image/} y que básicamente consiste en, situándonos en el subdirectorio donde hayamos descargado la imagen, teclear:

\begin{sphinxVerbatim}[commandchars=\\\{\},numbers=left,firstnumber=1,stepnumber=1]
\PYGZdl{}\PYG{+w}{ }xzcat\PYG{+w}{ }fichero\PYGZus{}imagen\PYGZus{}descargado.img.xz\PYG{+w}{ }\PYG{p}{|}\PYG{+w}{ }sudo\PYG{+w}{ }dd\PYG{+w}{ }\PYG{n+nv}{of}\PYG{o}{=}/dev/\PYG{o}{\PYGZob{}}indicador\PYG{+w}{ }de\PYG{+w}{ }la\PYG{+w}{ }tarjeta\PYG{+w}{ }SD\PYG{o}{\PYGZcb{}}\PYG{+w}{ }\PYG{n+nv}{bs}\PYG{o}{=}64k\PYG{+w}{ }\PYG{n+nv}{oflag}\PYG{o}{=}dsync\PYG{+w}{ }\PYG{n+nv}{status}\PYG{o}{=}progress
\end{sphinxVerbatim}

\begin{sphinxadmonition}{warning}{Advertencia:}
\sphinxAtStartPar
¡Asegúrate de tener el indicador correcto para la tarjeta SD! (¡corres el riesgo de perder el disco de tu sistema si no identificas bien la tarjeta del nuevo sistema operativo!)
\end{sphinxadmonition}

\sphinxAtStartPar
Una manera fácil de ver cual es el dispositivo al que está conectada nuestra tarjeta \sphinxstyleemphasis{microSD} es el siguiente:
\begin{enumerate}
\sphinxsetlistlabels{\alph}{enumi}{enumii}{}{.}%
\item {} 
\sphinxAtStartPar
Con la tarjeta \sphinxstyleemphasis{microSD} desconectada del ordenador teclear el comando \sphinxtitleref{lsblk} y observar que elementos hay en nuestro sistema debajo de \sphinxstyleemphasis{/dev/}. Por ejemplo en mi sistema obtengo lo siguiente:

\end{enumerate}

\begin{sphinxVerbatim}[commandchars=\\\{\},numbers=left,firstnumber=1,stepnumber=1]
\PYG{+w}{ }angel@debianHP:\PYGZti{}\PYGZdl{}\PYG{+w}{ }lsblk
\PYG{+w}{ }NAME\PYG{+w}{        }MAJ:MIN\PYG{+w}{ }RM\PYG{+w}{   }SIZE\PYG{+w}{ }RO\PYG{+w}{ }TYPE\PYG{+w}{ }MOUNTPOINT
\PYG{+w}{ }sda\PYG{+w}{           }\PYG{l+m}{8}:0\PYG{+w}{    }\PYG{l+m}{0}\PYG{+w}{   }\PYG{l+m}{1},8T\PYG{+w}{  }\PYG{l+m}{0}\PYG{+w}{ }disk
\PYG{+w}{ }├─sda1\PYG{+w}{        }\PYG{l+m}{8}:1\PYG{+w}{    }\PYG{l+m}{0}\PYG{+w}{   }\PYG{l+m}{1},8T\PYG{+w}{  }\PYG{l+m}{0}\PYG{+w}{ }part\PYG{+w}{ }/
\PYG{+w}{ }├─sda2\PYG{+w}{        }\PYG{l+m}{8}:2\PYG{+w}{    }\PYG{l+m}{0}\PYG{+w}{     }1K\PYG{+w}{  }\PYG{l+m}{0}\PYG{+w}{ }part
\PYG{+w}{ }└─sda5\PYG{+w}{        }\PYG{l+m}{8}:5\PYG{+w}{    }\PYG{l+m}{0}\PYG{+w}{   }975M\PYG{+w}{  }\PYG{l+m}{0}\PYG{+w}{ }part\PYG{+w}{ }\PYG{o}{[}SWAP\PYG{o}{]}
\PYG{+w}{ }nvme0n1\PYG{+w}{     }\PYG{l+m}{259}:0\PYG{+w}{    }\PYG{l+m}{0}\PYG{+w}{ }\PYG{l+m}{238},5G\PYG{+w}{  }\PYG{l+m}{0}\PYG{+w}{ }disk
\PYG{+w}{ }├─nvme0n1p1\PYG{+w}{ }\PYG{l+m}{259}:1\PYG{+w}{    }\PYG{l+m}{0}\PYG{+w}{   }260M\PYG{+w}{  }\PYG{l+m}{0}\PYG{+w}{ }part
\PYG{+w}{ }├─nvme0n1p2\PYG{+w}{ }\PYG{l+m}{259}:2\PYG{+w}{    }\PYG{l+m}{0}\PYG{+w}{    }16M\PYG{+w}{  }\PYG{l+m}{0}\PYG{+w}{ }part
\PYG{+w}{ }├─nvme0n1p3\PYG{+w}{ }\PYG{l+m}{259}:3\PYG{+w}{    }\PYG{l+m}{0}\PYG{+w}{ }\PYG{l+m}{237},2G\PYG{+w}{  }\PYG{l+m}{0}\PYG{+w}{ }part
\PYG{+w}{ }└─nvme0n1p4\PYG{+w}{ }\PYG{l+m}{259}:4\PYG{+w}{    }\PYG{l+m}{0}\PYG{+w}{   }980M\PYG{+w}{  }\PYG{l+m}{0}\PYG{+w}{ }part
\PYG{+w}{ }angel@debianHP:\PYGZti{}\PYGZdl{}
\end{sphinxVerbatim}
\begin{enumerate}
\sphinxsetlistlabels{\alph}{enumi}{enumii}{}{.}%
\setcounter{enumi}{1}
\item {} 
\sphinxAtStartPar
Y, a continuación, conectamos nuestra tarjeta \sphinxstyleemphasis{microSD}, repetimos el comando y vemos la diferencia,

\end{enumerate}

\begin{sphinxVerbatim}[commandchars=\\\{\},numbers=left,firstnumber=1,stepnumber=1]
\PYG{+w}{ }angel@debianHP:\PYGZti{}\PYGZdl{}\PYG{+w}{ }lsblk
\PYG{+w}{ }NAME\PYG{+w}{        }MAJ:MIN\PYG{+w}{ }RM\PYG{+w}{   }SIZE\PYG{+w}{ }RO\PYG{+w}{ }TYPE\PYG{+w}{ }MOUNTPOINT
\PYG{+w}{ }sda\PYG{+w}{           }\PYG{l+m}{8}:0\PYG{+w}{    }\PYG{l+m}{0}\PYG{+w}{   }\PYG{l+m}{1},8T\PYG{+w}{  }\PYG{l+m}{0}\PYG{+w}{ }disk
\PYG{+w}{ }├─sda1\PYG{+w}{        }\PYG{l+m}{8}:1\PYG{+w}{    }\PYG{l+m}{0}\PYG{+w}{   }\PYG{l+m}{1},8T\PYG{+w}{  }\PYG{l+m}{0}\PYG{+w}{ }part\PYG{+w}{ }/
\PYG{+w}{ }├─sda2\PYG{+w}{        }\PYG{l+m}{8}:2\PYG{+w}{    }\PYG{l+m}{0}\PYG{+w}{     }1K\PYG{+w}{  }\PYG{l+m}{0}\PYG{+w}{ }part
\PYG{+w}{ }└─sda5\PYG{+w}{        }\PYG{l+m}{8}:5\PYG{+w}{    }\PYG{l+m}{0}\PYG{+w}{   }975M\PYG{+w}{  }\PYG{l+m}{0}\PYG{+w}{ }part\PYG{+w}{ }\PYG{o}{[}SWAP\PYG{o}{]}
\PYG{+w}{ }mmcblk0\PYG{+w}{     }\PYG{l+m}{179}:0\PYG{+w}{    }\PYG{l+m}{0}\PYG{+w}{  }\PYG{l+m}{29},7G\PYG{+w}{  }\PYG{l+m}{0}\PYG{+w}{ }disk
\PYG{+w}{ }└─mmcblk0p1\PYG{+w}{ }\PYG{l+m}{179}:1\PYG{+w}{    }\PYG{l+m}{0}\PYG{+w}{  }\PYG{l+m}{29},7G\PYG{+w}{  }\PYG{l+m}{0}\PYG{+w}{ }part\PYG{+w}{ }/media/angel/ext4\PYGZhy{}MICROSD
\PYG{+w}{ }nvme0n1\PYG{+w}{     }\PYG{l+m}{259}:0\PYG{+w}{    }\PYG{l+m}{0}\PYG{+w}{ }\PYG{l+m}{238},5G\PYG{+w}{  }\PYG{l+m}{0}\PYG{+w}{ }disk
\PYG{+w}{ }├─nvme0n1p1\PYG{+w}{ }\PYG{l+m}{259}:1\PYG{+w}{    }\PYG{l+m}{0}\PYG{+w}{   }260M\PYG{+w}{  }\PYG{l+m}{0}\PYG{+w}{ }part
\PYG{+w}{ }├─nvme0n1p2\PYG{+w}{ }\PYG{l+m}{259}:2\PYG{+w}{    }\PYG{l+m}{0}\PYG{+w}{    }16M\PYG{+w}{  }\PYG{l+m}{0}\PYG{+w}{ }part
\PYG{+w}{ }├─nvme0n1p3\PYG{+w}{ }\PYG{l+m}{259}:3\PYG{+w}{    }\PYG{l+m}{0}\PYG{+w}{ }\PYG{l+m}{237},2G\PYG{+w}{  }\PYG{l+m}{0}\PYG{+w}{ }part
\PYG{+w}{ }└─nvme0n1p4\PYG{+w}{ }\PYG{l+m}{259}:4\PYG{+w}{    }\PYG{l+m}{0}\PYG{+w}{   }980M\PYG{+w}{  }\PYG{l+m}{0}\PYG{+w}{ }part
\PYG{+w}{ }angel@debianHP:\PYGZti{}\PYGZdl{}
\end{sphinxVerbatim}

\sphinxAtStartPar
Vemos que el dispositivo que hay que utilizar es \sphinxtitleref{/dev/mmcblk0}. Podéis ver que yo tenía formateada la tarjeta \sphinxstyleemphasis{microSD} como ext4, como indica la etiqueta con la que la creé.

\begin{sphinxadmonition}{note}{Nota:}
\sphinxAtStartPar
Puede ser conveniente que, si la tarjeta \sphinxstyleemphasis{microSD} no es nueva, la formatees previamente. Lo puedes hacer con un interfaz gráfico con un programa como \sphinxstyleemphasis{GParted}, o desde la linea de comandos tal como se muestra a continuación. Si tienes instalados los paquetes \sphinxstyleemphasis{parted} \sphinxstyleemphasis{dosfstools} y \sphinxstyleemphasis{mtools} puedes ignorar las dos primeras líneas.
\end{sphinxadmonition}

\begin{sphinxVerbatim}[commandchars=\\\{\},numbers=left,firstnumber=1,stepnumber=1]
\PYGZdl{}\PYG{+w}{ }sudo\PYG{+w}{ }apt\PYG{+w}{ }update
\PYGZdl{}\PYG{+w}{ }sudo\PYG{+w}{ }apt\PYG{+w}{ }install\PYG{+w}{ }parted\PYG{+w}{ }dosfstools\PYG{+w}{ }mtools
\PYGZdl{}\PYG{+w}{ }sudo\PYG{+w}{ }umount\PYG{+w}{ }/media/angel/ext4\PYGZhy{}MICROSD
\PYGZdl{}\PYG{+w}{ }sudo\PYG{+w}{ }parted\PYG{+w}{ }/dev/mmcblk0\PYG{+w}{ }\PYGZhy{}\PYGZhy{}script\PYG{+w}{ }\PYGZhy{}\PYGZhy{}\PYG{+w}{ }mklabel\PYG{+w}{ }msdos
\PYGZdl{}\PYG{+w}{ }sudo\PYG{+w}{ }parted\PYG{+w}{ }/dev/mmcblk0\PYG{+w}{ }\PYGZhy{}\PYGZhy{}script\PYG{+w}{ }\PYGZhy{}\PYGZhy{}\PYG{+w}{ }mkpart\PYG{+w}{ }primary\PYG{+w}{ }fat32\PYG{+w}{ }1MiB\PYG{+w}{ }\PYG{l+m}{100}\PYGZpc{}
\PYGZdl{}\PYG{+w}{ }sudo\PYG{+w}{ }mkfs.vfat\PYG{+w}{ }\PYGZhy{}F32\PYG{+w}{ }/dev/mmcblk0
\PYGZdl{}\PYG{+w}{ }sudo\PYG{+w}{ }mlabel\PYG{+w}{ }\PYGZhy{}i\PYG{+w}{ }/dev/mmcblk0\PYG{+w}{ }::mietiqueta
\end{sphinxVerbatim}


\subsection{Conexión a Internet}
\label{\detokenize{DebianNativoRPi:conexion-a-internet}}
\sphinxAtStartPar
Si tu tarjeta tiene conector RJ45 para conectar una línea ethernet puedes conectarte al router con el cable. La conexión a internet será sencilla y no hay que hacer nada. Si no hay conector RJ45 o solo puedes conectar por \sphinxstyleemphasis{wifi} es necesario hacer dos cosas más en la \sphinxstyleemphasis{microSD} antes de ponerla en el zócalo de la tarjeta \sphinxstyleemphasis{Raspberry Pi}.

\sphinxAtStartPar
La tarjeta que acabamos de crear tiene dos particiones: \sphinxstyleemphasis{RASPIFIRM} donde están todos los ficheros para el arranque del Sistema Operativo y \sphinxstyleemphasis{RASPIROOT} que contiene la estructura de ficheros de nuestro Sistema \sphinxstyleemphasis{Debian}.

\sphinxAtStartPar
Para tener conexión a \sphinxstyleemphasis{Internet} debemos editar el fichero \sphinxstyleemphasis{/etc/network/interfaces.d/wlan0} que está en la partición \sphinxstyleemphasis{RASPIROOT} de la tarjeta que acabamos de crear. El contenido del fichero deberá ser como el siguiente

\begin{sphinxVerbatim}[commandchars=\\\{\},numbers=left,firstnumber=1,stepnumber=1]
\PYG{c+c1}{\PYGZsh{} To enable wireless networking, uncomment the following lines and \PYGZhy{}naturally\PYGZhy{}}
\PYG{c+c1}{\PYGZsh{} replace with your network\PYGZsq{}s details.}
\PYG{c+c1}{\PYGZsh{}}
allow\PYGZhy{}hotplug\PYG{+w}{ }wlan0
iface\PYG{+w}{ }wlan0\PYG{+w}{ }inet\PYG{+w}{ }dhcp
\PYG{c+c1}{\PYGZsh{}iface wlan0 inet6 dhcp}
\PYG{+w}{  }wpa\PYGZhy{}ssid\PYG{+w}{ }\PYG{l+s+s2}{\PYGZdq{}tu\PYGZus{}identificador\PYGZus{}de\PYGZus{}red\PYGZdq{}}
\PYG{+w}{  }wpa\PYGZhy{}psk\PYG{+w}{ }\PYG{l+s+s2}{\PYGZdq{}tu\PYGZus{}contraseña\PYGZdq{}}
\end{sphinxVerbatim}

\sphinxAtStartPar
En la línea 5 o 6 deberás descomentar la correspondiente a tu router: descomenta la línea 5 si tienes direccionamiento \sphinxstyleemphasis{IPv4} (del estilo 192.168.1.23) o descomenta la línea 6 si tienes direccionamiento \sphinxstyleemphasis{IPv6} (del estilo 2001:0db8:85a3:0000:0000:8a2e:0370:7334).

\sphinxAtStartPar
En las líneas 7 y 8 deberás escribir los datos de tu red.


\subsection{Conexión a la tarjeta \sphinxstyleemphasis{Raspberry Pi} sin monitor}
\label{\detokenize{DebianNativoRPi:conexion-a-la-tarjeta-raspberry-pi-sin-monitor}}
\sphinxAtStartPar
Si no dispones de monitor para conectar a tu tarjeta \sphinxstyleemphasis{Raspberry Pi} puedes conectarte desde tu ordenador (el que has utilizado para grabar la imagen) mediante \sphinxstyleemphasis{ssh}. Para ello debes modificar el fichero \sphinxstyleemphasis{/boot/firmware/sysconf.txt} que está en la partición \sphinxstyleemphasis{RASPIFIRM}. Previamente has de generar la clave de autorización para poder acceder mediante \sphinxstyleemphasis{ssh}. Para ello hay que generar unos ficheros (\sphinxstyleemphasis{id\_rsaP} y \sphinxstyleemphasis{id\_rsa.pub}) que se crean en tu subdirectorio raiz bajo \sphinxstyleemphasis{/home/usuario/.ssh} mediante el comando:

\begin{sphinxVerbatim}[commandchars=\\\{\},numbers=left,firstnumber=1,stepnumber=1]
ssh\PYGZhy{}keygen\PYG{+w}{ }\PYGZhy{}t\PYG{+w}{ }rsa
\end{sphinxVerbatim}

\sphinxAtStartPar
Cuando lo ejecutes te pedirá una \sphinxstyleemphasis{passphrase} que es una contraseña (nos recomienda que sea una frase larga) que deberás recordar para poder utilizar el \sphinxstyleemphasis{ssh}. Si lo dejas en blanco no te pédirá esta contraseña pero el sistema estará más desprotegido.

\sphinxAtStartPar
Una vez ejecutado este comando debemos copiar el contenido del fichero \sphinxstyleemphasis{id\_rsa.pub}, que debe ser del estilo de lo que ves a continuación:

\begin{sphinxVerbatim}[commandchars=\\\{\},numbers=left,firstnumber=1,stepnumber=1]
\PYG{+w}{ }ssh\PYGZhy{}rsa\PYG{+w}{ }AAAAB3NzaC1yc2EBAAADAQABAAABgQCyuJLQDeYhvitA4rCS9xRgBEf2L+sCsadGScC2H2VJN0//G3K0W57IX58t2ptaJunfEakxxZ+fperXLFIDBQaeDLQyQXwI8CA/Dj4DRjQMh9MuRRWMBrTMYz0rdIc0bnlHVh5q9SgJJReyfs4sibEV4wZUq3GIe+8lqGlpbGqevDdU/TY6swHBs8Ff+N187xHCvd6NJCBYYDWddnUSj4WPdsOdUgiGMpmXII5M9zGqztycAuwpeMeW13L2GBCEI8q6pqUWUUwiOMNsPN/BVFqsnrnXxBFVMS7CSGkFCeUwvQDz9LY2gHbs9x9lJWLT0D9dg2YkCuGDBe/H0k7EdsH3FXsO79YdUfnA91yWe81oGlBwIYqo2+bAx91sbc/DJhML9G0UsxsHGC/88BFeaQ8jmwoi21x0ZBNOkfAQaR5u1cSZYE5340cy5/FZvF5PWiV7XZQuVS6VeJg2H9n3il/APAuB34XjIyvr/hzJrASr4YtNkAtKi2FlIxntUkAYbmk\PYG{o}{=}\PYG{+w}{ }usuario@debian
\end{sphinxVerbatim}

\sphinxAtStartPar
en el fichero \sphinxstyleemphasis{/boot/firmware/sysconf.txt} y descomentando la línea \sphinxstyleemphasis{root\_autorized\_key=} de la partición \sphinxstyleemphasis{RASPIFIRM} a partir del signo \sphinxstyleemphasis{=} y sin espacio y que debe quedar del estilo a:

\begin{sphinxVerbatim}[commandchars=\\\{\},numbers=left,firstnumber=1,stepnumber=1]
\PYG{+w}{ }\PYG{n+nv}{root\PYGZus{}autorized\PYGZus{}key}\PYG{o}{=}ssh\PYGZhy{}rsa\PYG{+w}{ }AAAAB3NzaC1yc2EBAAADAQABAAABgQCyuJLQDeYhvitA4rCS9xRgBEf2L+sCsadGScC2H2VJN0//G3K0W57IX58t2ptaJunfEakxxZ+fperXLFIDBQaeDLQyQXwI8CA/Dj4DRjQMh9MuRRWMBrTMYz0rdIc0bnlHVh5q9SgJJReyfs4sibEV4wZUq3GIe+8lqGlpbGqevDdU/TY6swHBs8Ff+N187xHCvd6NJCBYYDWddnUSj4WPdsOdUgiGMpmXII5M9zGqztycAuwpeMeW13L2GBCEI8q6pqUWUUwiOMNsPN/BVFqsnrnXxBFVMS7CSGkFCeUwvQDz9LY2gHbs9x9lJWLT0D9dg2YkCuGDBe/H0k7EdsH3FXsO79YdUfnA91yWe81oGlBwIYqo2+bAx91sbc/DJhML9G0UsxsHGC/88BFeaQ8jmwoi21x0ZBNOkfAQaR5u1cSZYE5340cy5/FZvF5PWiV7XZQuVS6VeJg2H9n3il/APAuB34XjIyvr/hzJrASr4YtNkAtKi2FlIxntUkAYbmk\PYG{o}{=}\PYG{+w}{ }usuario@debian
\end{sphinxVerbatim}

\sphinxAtStartPar
Además debemos conocer las direcciones \sphinxstyleemphasis{IP} de nuestra red. Para ello es necesario tener instalado el programa \sphinxstyleemphasis{nmap}. Si no lo tienes instalado lo puedes instalar con:

\begin{sphinxVerbatim}[commandchars=\\\{\},numbers=left,firstnumber=1,stepnumber=1]
\PYGZdl{}\PYG{+w}{ }sudo\PYG{+w}{ }apt\PYG{+w}{ }install\PYG{+w}{ }nmap
\end{sphinxVerbatim}

\sphinxAtStartPar
Alimentamos la tarjeta \sphinxstyleemphasis{Raspberry Pi} con la \sphinxstyleemphasis{microSD} en su zócalo y esperamos unos minutos para proseguir. La \sphinxstyleemphasis{Raspberry Pi} puede tardar unos minutos en arrancar completamente.

\sphinxAtStartPar
Para poder conectarnos con la tarjeta \sphinxstyleemphasis{Raspberry Pi} utilizamos el comando \sphinxstyleemphasis{hostname \sphinxhyphen{}I} para saber cual es la dirección \sphinxstyleemphasis{IP} de nuestro \sphinxstyleemphasis{PC} y una vez identificado utilizamos \sphinxstyleemphasis{nmap} con la dirección de nuestro \sphinxstyleemphasis{PC} pero en el rango \sphinxstyleemphasis{1\sphinxhyphen{}255} para descubrir la dirección de nuestra \sphinxstyleemphasis{Raspberry Pi}.

\begin{sphinxVerbatim}[commandchars=\\\{\},numbers=left,firstnumber=1,stepnumber=1]
\PYGZdl{}\PYG{+w}{ }hostname\PYG{+w}{ }\PYGZhy{}I
\PYG{l+m}{192}.168.1.105
\PYGZdl{}\PYG{+w}{ }nmap\PYG{+w}{ }\PYGZhy{}sP\PYG{+w}{ }\PYG{l+m}{192}.168.1.1\PYGZhy{}255
Starting\PYG{+w}{ }Nmap\PYG{+w}{ }\PYG{l+m}{7}.93\PYG{+w}{ }\PYG{o}{(}\PYG{+w}{ }https://nmap.org\PYG{+w}{ }\PYG{o}{)}\PYG{+w}{ }at\PYG{+w}{ }\PYG{l+m}{2023}\PYGZhy{}09\PYGZhy{}13\PYG{+w}{ }\PYG{l+m}{17}:17\PYG{+w}{ }CEST
Nmap\PYG{+w}{ }scan\PYG{+w}{ }report\PYG{+w}{ }\PYG{k}{for}\PYG{+w}{ }mi\PYGZus{}rouiter\PYG{+w}{ }\PYG{o}{(}\PYG{l+m}{192}.168.1.1\PYG{o}{)}
Host\PYG{+w}{ }is\PYG{+w}{ }up\PYG{+w}{ }\PYG{o}{(}\PYG{l+m}{0}.0014s\PYG{+w}{ }latency\PYG{o}{)}.
Nmap\PYG{+w}{ }scan\PYG{+w}{ }report\PYG{+w}{ }\PYG{k}{for}\PYG{+w}{ }rpi1\PYGZhy{}20230908.home\PYG{+w}{ }\PYG{o}{(}\PYG{l+m}{192}.168.1.106\PYG{o}{)}
Host\PYG{+w}{ }is\PYG{+w}{ }up\PYG{+w}{ }\PYG{o}{(}\PYG{l+m}{0}.048s\PYG{+w}{ }latency\PYG{o}{)}.
Nmap\PYG{+w}{ }scan\PYG{+w}{ }report\PYG{+w}{ }\PYG{k}{for}\PYG{+w}{ }debHP5T.home\PYG{+w}{ }\PYG{o}{(}\PYG{l+m}{192}.168.1.105\PYG{o}{)}
Host\PYG{+w}{ }is\PYG{+w}{ }up\PYG{+w}{ }\PYG{o}{(}\PYG{l+m}{0}.00073s\PYG{+w}{ }latency\PYG{o}{)}.
Nmap\PYG{+w}{ }\PYG{k}{done}:\PYG{+w}{ }\PYG{l+m}{255}\PYG{+w}{ }IP\PYG{+w}{ }addresses\PYG{+w}{ }\PYG{o}{(}\PYG{l+m}{3}\PYG{+w}{ }hosts\PYG{+w}{ }up\PYG{o}{)}\PYG{+w}{ }scanned\PYG{+w}{ }\PYG{k}{in}\PYG{+w}{ }\PYG{l+m}{5}.37\PYG{+w}{ }seconds
\end{sphinxVerbatim}

\sphinxAtStartPar
En este ejemplo la \sphinxstyleemphasis{Raspberry Pi} está en la dirección \sphinxstyleemphasis{192.168.1.106}.

\sphinxAtStartPar
Para conectar con la \sphinxstyleemphasis{Raspberry Pi} tecleamos lo siguiente (en este paso nos pedirá la \sphinxstyleemphasis{passphrase} si es que la creamos anteriormente):

\begin{sphinxVerbatim}[commandchars=\\\{\},numbers=left,firstnumber=1,stepnumber=1]
\PYGZdl{} ssh root@192.168.1.106
The authenticity of host \PYGZsq{}192.168.1.106 (192.168.1.106)\PYGZsq{} can\PYGZsq{}t be established.
ED25519 key fingerprint is SHA256:iG2kEuxjKaxgRYdF7gj3den/J0NsNM7fPoe/ZkFbskM.
This key is not known by any other names.
Are you sure you want to continue connecting (yes/no/[fingerprint])?
\end{sphinxVerbatim}

\sphinxAtStartPar
Como vemos nos pide expresamente que confirmemos con \sphinxstyleemphasis{yes}. Tecleamos \sphinxstyleemphasis{yes} y \sphinxstyleemphasis{Enter} y veremos algo similar a:

\begin{sphinxVerbatim}[commandchars=\\\{\},numbers=left,firstnumber=1,stepnumber=1]
Are\PYG{+w}{ }you\PYG{+w}{ }sure\PYG{+w}{ }you\PYG{+w}{ }want\PYG{+w}{ }to\PYG{+w}{ }\PYG{k}{continue}\PYG{+w}{ }connecting\PYG{+w}{ }\PYG{o}{(}yes/no/\PYG{o}{[}fingerprint\PYG{o}{]}\PYG{o}{)}?\PYG{+w}{ }yes
Warning:\PYG{+w}{ }Permanently\PYG{+w}{ }added\PYG{+w}{ }\PYG{l+s+s1}{\PYGZsq{}192.168.1.106\PYGZsq{}}\PYG{+w}{ }\PYG{o}{(}ED25519\PYG{o}{)}\PYG{+w}{ }to\PYG{+w}{ }the\PYG{+w}{ }list\PYG{+w}{ }of\PYG{+w}{ }known\PYG{+w}{ }hosts.
Connection\PYG{+w}{ }closed\PYG{+w}{ }by\PYG{+w}{ }\PYG{l+m}{192}.168.1.106\PYG{+w}{ }port\PYG{+w}{ }\PYG{l+m}{22}
\end{sphinxVerbatim}

\sphinxAtStartPar
Si ya tuviéramos el usuario con password (cuando tengamos el \sphinxstyleemphasis{S.O.} instalado y operativo), la respuesta sería similar a:

\begin{sphinxVerbatim}[commandchars=\\\{\},numbers=left,firstnumber=1,stepnumber=1]
\PYGZdl{}\PYG{+w}{ }ssh\PYG{+w}{ }usuario@192.168.1.106
The\PYG{+w}{ }authenticity\PYG{+w}{ }of\PYG{+w}{ }host\PYG{+w}{ }\PYG{l+s+s1}{\PYGZsq{}192.168.1.106 (192.168.1.106)\PYGZsq{}}\PYG{+w}{ }can\PYG{l+s+s1}{\PYGZsq{}t be established.}
\PYG{l+s+s1}{ED25519 key fingerprint is SHA256:iG2kEuxjKaxgRYdF7gj3den/J0NsNM7fPoe/ZkFbskM.}
\PYG{l+s+s1}{This key is not known by any other names.}
\PYG{l+s+s1}{Are you sure you want to continue connecting (yes/no/[fingerprint])? yes}
\PYG{l+s+s1}{Warning: Permanently added \PYGZsq{}}\PYG{l+m}{192}.168.1.106\PYG{l+s+s1}{\PYGZsq{} (ED25519) to the list of known hosts.}
\PYG{l+s+s1}{usuario@192.168.1.106\PYGZsq{}}s\PYG{+w}{ }password:
Linux\PYG{+w}{ }rpi1\PYGZhy{}20230908\PYG{+w}{ }\PYG{l+m}{6}.1.0\PYGZhy{}11\PYGZhy{}rpi\PYG{+w}{ }\PYG{c+c1}{\PYGZsh{}1 Debian 6.1.38\PYGZhy{}4 (2023\PYGZhy{}08\PYGZhy{}08) armv6l}

The\PYG{+w}{ }programs\PYG{+w}{ }included\PYG{+w}{ }with\PYG{+w}{ }the\PYG{+w}{ }Debian\PYG{+w}{ }GNU/Linux\PYG{+w}{ }system\PYG{+w}{ }are\PYG{+w}{ }free\PYG{+w}{ }software\PYG{p}{;}
the\PYG{+w}{ }exact\PYG{+w}{ }distribution\PYG{+w}{ }terms\PYG{+w}{ }\PYG{k}{for}\PYG{+w}{ }each\PYG{+w}{ }program\PYG{+w}{ }are\PYG{+w}{ }described\PYG{+w}{ }\PYG{k}{in}\PYG{+w}{ }the
individual\PYG{+w}{ }files\PYG{+w}{ }\PYG{k}{in}\PYG{+w}{ }/usr/share/doc/*/copyright.

Debian\PYG{+w}{ }GNU/Linux\PYG{+w}{ }comes\PYG{+w}{ }with\PYG{+w}{ }ABSOLUTELY\PYG{+w}{ }NO\PYG{+w}{ }WARRANTY,\PYG{+w}{ }to\PYG{+w}{ }the\PYG{+w}{ }extent
permitted\PYG{+w}{ }by\PYG{+w}{ }applicable\PYG{+w}{ }law.
Last\PYG{+w}{ }login:\PYG{+w}{ }Thu\PYG{+w}{ }Sep\PYG{+w}{ }\PYG{l+m}{14}\PYG{+w}{ }\PYG{l+m}{09}:28:33\PYG{+w}{ }\PYG{l+m}{2023}\PYG{+w}{ }from\PYG{+w}{ }\PYG{l+m}{192}.168.1.28
\end{sphinxVerbatim}

\begin{sphinxadmonition}{note}{Nota:}
\sphinxAtStartPar
Inicialmente el usuario root no tiene \sphinxstyleemphasis{password}
\end{sphinxadmonition}

\sphinxAtStartPar
Y ya puedes empezar a trabajar con un monitor remoto.

\begin{sphinxadmonition}{note}{Nota:}
\sphinxAtStartPar
Cuando tengas instalado el sistema completo podrás conectarte mediante \sphinxstyleemphasis{ssh} con un ordenador (como estamos haciendo ahora) pero también lo puedes hacer con una tablet o un teléfono móvil.
\end{sphinxadmonition}

\sphinxAtStartPar
Cuando hayas creado un usuario el comando para la conexión será:

\begin{sphinxVerbatim}[commandchars=\\\{\},numbers=left,firstnumber=1,stepnumber=1]
\PYGZdl{} ssh usuario@192.168.1.106
usuario@192.168.1.106\PYGZsq{}s password:
\end{sphinxVerbatim}

\sphinxAtStartPar
Donde \sphinxstyleemphasis{usuario} será en nombre del usuario que hayas creado.


\subsection{Configuración del sistema operativo}
\label{\detokenize{DebianNativoRPi:configuracion-del-sistema-operativo}}
\sphinxAtStartPar
Una vez grabado el \sphinxstyleemphasis{SO} en la \sphinxstyleemphasis{microSD} la insertamos en nuestra \sphinxstyleemphasis{RPi} y esperamos a que aparezca el \sphinxstyleemphasis{login} de entrada al sistema. Entramos con el usuario \sphinxstyleemphasis{root} y veremos que \sphinxstylestrong{no} nos pide contraseña. Los pasos para configurar nuestro sistema en español son los siguientes:
\begin{enumerate}
\sphinxsetlistlabels{\arabic}{enumi}{enumii}{}{.}%
\item {} 
\sphinxAtStartPar
Actualización del sistema

\end{enumerate}

\begin{sphinxVerbatim}[commandchars=\\\{\},numbers=left,firstnumber=1,stepnumber=1]
\PYG{c+c1}{\PYGZsh{} apt update}
\PYG{c+c1}{\PYGZsh{} apt upgrade}
\end{sphinxVerbatim}

\begin{sphinxadmonition}{note}{Nota:}
\sphinxAtStartPar
La primera vez que tecleas \sphinxstyleemphasis{apt update} el reloj del sistema no está sincronizado y produce un error de repositorio antiguo. La segunda vez es posible que ya se haya sincronizado y ya lo acepte sin error. Puedes hacer \sphinxstyleemphasis{ping google.es} y ver si tienes o no respuesta.
\end{sphinxadmonition}

\sphinxAtStartPar
Si al hacer la actualización (\sphinxstyleemphasis{apt upgrade}) hace preguntas contesta la respuesta por defecto (\sphinxstyleemphasis{Intro}).
\begin{enumerate}
\sphinxsetlistlabels{\arabic}{enumi}{enumii}{}{.}%
\setcounter{enumi}{1}
\item {} 
\sphinxAtStartPar
Añadir una \sphinxstyleemphasis{password} para el usuario \sphinxstyleemphasis{root}.

\end{enumerate}

\begin{sphinxVerbatim}[commandchars=\\\{\},numbers=left,firstnumber=1,stepnumber=1]
\PYG{c+c1}{\PYGZsh{} passwd}
New\PYG{+w}{ }password:\PYG{+w}{ }\PYG{l+s+s1}{\PYGZsq{}Tu\PYGZus{}password\PYGZus{}para\PYGZus{}root\PYGZsq{}}
Retype\PYG{+w}{ }new\PYG{+w}{ }password:\PYG{+w}{ }\PYG{l+s+s1}{\PYGZsq{}Tu\PYGZus{}password\PYGZus{}para\PYGZus{}root\PYGZsq{}}
\end{sphinxVerbatim}
\begin{enumerate}
\sphinxsetlistlabels{\arabic}{enumi}{enumii}{}{.}%
\setcounter{enumi}{2}
\item {} 
\sphinxAtStartPar
Instalamos el programa \sphinxstyleemphasis{sudo} que permite a un usuario normal tener los privilegios de \sphinxstyleemphasis{root}.

\end{enumerate}

\begin{sphinxVerbatim}[commandchars=\\\{\},numbers=left,firstnumber=1,stepnumber=1]
\PYG{c+c1}{\PYGZsh{} apt install sudo}
\end{sphinxVerbatim}
\begin{enumerate}
\sphinxsetlistlabels{\arabic}{enumi}{enumii}{}{.}%
\setcounter{enumi}{3}
\item {} 
\sphinxAtStartPar
Creamos un usuario nuevo. Yo voy a crear, como ejemplo, el usuario \sphinxstyleemphasis{usuario}. Te pedirá el nombre del usuario y su contraseña. El resto de los campos que solicita son opcionales (yo los dejo en blanco pulsando \sphinxstyleemphasis{Intro}) y al final pedirá confirmación.

\end{enumerate}

\begin{sphinxVerbatim}[commandchars=\\\{\},numbers=left,firstnumber=1,stepnumber=1]
\PYG{c+c1}{\PYGZsh{} adduser usuario}
\end{sphinxVerbatim}

\begin{sphinxadmonition}{note}{Nota:}
\sphinxAtStartPar
En usuario pon el nombre que quieres para ti en el sistema.
\end{sphinxadmonition}
\begin{enumerate}
\sphinxsetlistlabels{\arabic}{enumi}{enumii}{}{.}%
\setcounter{enumi}{4}
\item {} 
\sphinxAtStartPar
Para que \sphinxstyleemphasis{usuario} pertenezca al grupo \sphinxstyleemphasis{sudo} hacemos lo siguiente:

\end{enumerate}

\begin{sphinxVerbatim}[commandchars=\\\{\},numbers=left,firstnumber=1,stepnumber=1]
\PYG{c+c1}{\PYGZsh{} usermod \PYGZhy{}aG sudo usuario}
\end{sphinxVerbatim}

\sphinxAtStartPar
Aquí tenemos un problema. Todavía no tenemos instalado el teclado y las \sphinxstyleemphasis{Locales} en español, por lo que el guión «\sphinxhyphen{}» no está en la tecla de nuestro teclado. Podemos ver en \sphinxurl{https://es.wikipedia.org/wiki/Distribuci\%C3\%B3n\_del\_teclado} que el guión en el teclado de Estados Unidos está en la tercera tecla, por la derecha, de la fila de números y símbolos del teclado y que en el teclado español se corresponde con la tecla «\textquotesingle{}» (comilla simple) la que tiene el «?» cuando pulsamos la tecla \sphinxstyleemphasis{Shift} o \sphinxstyleemphasis{Mayúsculas}.
\begin{enumerate}
\sphinxsetlistlabels{\arabic}{enumi}{enumii}{}{.}%
\setcounter{enumi}{5}
\item {} 
\sphinxAtStartPar
Ahora podemos reiniciar el sistema y entrar como usuario \sphinxstyleemphasis{root} o \sphinxstyleemphasis{usuario} con su contraseña correspondiente.

\end{enumerate}

\begin{sphinxVerbatim}[commandchars=\\\{\},numbers=left,firstnumber=1,stepnumber=1]
\PYG{c+c1}{\PYGZsh{} reboot}
\end{sphinxVerbatim}

\sphinxAtStartPar
Cuando arranque de nuevo entramos como usuario \sphinxstyleemphasis{usuario} y su contraseña

\begin{sphinxVerbatim}[commandchars=\\\{\},numbers=left,firstnumber=1,stepnumber=1]
login:\PYG{+w}{ }usuario
Password:
\end{sphinxVerbatim}
\begin{enumerate}
\sphinxsetlistlabels{\arabic}{enumi}{enumii}{}{.}%
\setcounter{enumi}{6}
\item {} 
\sphinxAtStartPar
A continuación instalamos las \sphinxstyleemphasis{locales}. Al utilizar \sphinxstyleemphasis{sudo} nos pedirá la password de \sphinxstyleemphasis{usuario} para proceder.

\end{enumerate}

\begin{sphinxVerbatim}[commandchars=\\\{\},numbers=left,firstnumber=1,stepnumber=1]
\PYGZdl{}\PYG{+w}{ }sudo\PYG{+w}{ }apt\PYG{+w}{ }install\PYG{+w}{ }locales
\PYGZdl{}\PYG{+w}{ }sudo\PYG{+w}{ }dpkg\PYGZhy{}reconfigure\PYG{+w}{ }locales
\end{sphinxVerbatim}

\sphinxAtStartPar
Y seleccionaremos,  con \sphinxstyleemphasis{la barra de espacio}, dos: \sphinxstyleemphasis{en\_US.UTF8 UTF\sphinxhyphen{}8} y \sphinxstyleemphasis{es\_ES.UTF8 UTF\sphinxhyphen{}8}. Pulsamos \sphinxstyleemphasis{\textless{}tabular\textgreater{}} y \sphinxstyleemphasis{ok} y cuando nos pregunte que \sphinxstyleemphasis{locale} queremos que sea nuestra local por defecto seleccionamos \sphinxstyleemphasis{es\_ES.UTF\sphinxhyphen{}8}. Con esto el teclado todavía no está configurado en español.
\begin{enumerate}
\sphinxsetlistlabels{\arabic}{enumi}{enumii}{}{.}%
\setcounter{enumi}{7}
\item {} 
\sphinxAtStartPar
Configuración del teclado en español. Para ello hacemos lo siguiente:

\end{enumerate}

\begin{sphinxVerbatim}[commandchars=\\\{\},numbers=left,firstnumber=1,stepnumber=1]
\PYGZdl{}\PYG{+w}{ }sudo\PYG{+w}{ }apt\PYG{+w}{ }install\PYG{+w}{ }keyboard\PYGZhy{}configuration
\end{sphinxVerbatim}

\begin{sphinxadmonition}{note}{Nota:}
\sphinxAtStartPar
Todavía no tenemos configurado el teclado en español por lo que deberemos usar nuevamente la tecla «\textquotesingle{}».
\end{sphinxadmonition}

\sphinxAtStartPar
En la primera pantalla seleccionamos \sphinxstyleemphasis{Other}, pulsamos \sphinxstyleemphasis{\textless{}tabular\textgreater{}}, \sphinxstyleemphasis{ok} e \sphinxstyleemphasis{Intro}. De la lista que aparece seleccionamos \sphinxstyleemphasis{Spanish} y \sphinxstyleemphasis{ok} y, luego, \sphinxstyleemphasis{Spanish \sphinxhyphen{} Spanish (Windows)} y \sphinxstyleemphasis{ok}.
\begin{enumerate}
\sphinxsetlistlabels{\arabic}{enumi}{enumii}{}{.}%
\setcounter{enumi}{8}
\item {} 
\sphinxAtStartPar
Ahora volvemos a reiniciar el sistema y entrar como usuario \sphinxstyleemphasis{usuario} con su contraseña correspondiente.

\end{enumerate}

\begin{sphinxVerbatim}[commandchars=\\\{\},numbers=left,firstnumber=1,stepnumber=1]
\PYGZdl{}\PYG{+w}{ }sudo\PYG{+w}{ }reboot
\end{sphinxVerbatim}
\begin{enumerate}
\sphinxsetlistlabels{\arabic}{enumi}{enumii}{}{.}%
\setcounter{enumi}{9}
\item {} 
\sphinxAtStartPar
A continuación instalamos los paquetes necesarios para la consola:

\end{enumerate}

\begin{sphinxVerbatim}[commandchars=\\\{\},numbers=left,firstnumber=1,stepnumber=1]
\PYGZdl{}\PYG{+w}{ }sudo\PYG{+w}{ }apt\PYG{+w}{ }install\PYG{+w}{ }gpm\PYG{+w}{ }console\PYGZhy{}common\PYG{+w}{ }console\PYGZhy{}data\PYG{+w}{ }console\PYGZhy{}setup
\end{sphinxVerbatim}

\begin{sphinxadmonition}{note}{Nota:}
\sphinxAtStartPar
El paquete \sphinxstyleemphasis{gpm} es para poder usar el ratón en la consola.
\end{sphinxadmonition}

\sphinxAtStartPar
Veremos que ahora la \sphinxstyleemphasis{Configuración de console\sphinxhyphen{}data} ya nos aparece en español, aunque el teclado todavía no funciona en español.

\sphinxAtStartPar
Seleccionamos la opción \sphinxstyleemphasis{Elegir el mapa de teclado de la lista completa} y seleccionamos \sphinxstyleemphasis{\textless{}Aceptar\textgreater{}} y seleccionamos \sphinxstyleemphasis{pc / qwerty Spanish / Standard / Standard} y  \sphinxstyleemphasis{\textless{}Aceptar\textgreater{}} ya tendremos el teclado en español. Verás, además, que si mueves el ratón el cursor se moverá por la pantalla.
\begin{enumerate}
\sphinxsetlistlabels{\arabic}{enumi}{enumii}{}{.}%
\setcounter{enumi}{10}
\item {} 
\sphinxAtStartPar
Este paso es opcional. En la consola el tipo de caracteres (\sphinxstyleemphasis{fuentes}) que se han instalado es \sphinxstyleemphasis{Fixed} que tiene una mejor cobertura para los \sphinxstyleemphasis{scripts} internacionales. A mí, particularmente me gusta más los tipos \sphinxstyleemphasis{VGA}. En cualquier caso si deseas configurar los tipos de caracteres de la consola puedes hacer:

\end{enumerate}

\begin{sphinxVerbatim}[commandchars=\\\{\},numbers=left,firstnumber=1,stepnumber=1]
\PYGZdl{}\PYG{+w}{ }sudo\PYG{+w}{ }dpkg\PYGZhy{}reconfigure\PYG{+w}{ }console\PYGZhy{}settup
\end{sphinxVerbatim}

\begin{sphinxadmonition}{note}{Nota:}
\sphinxAtStartPar
Ahora ya sí, el guión «\sphinxhyphen{}» está en la tecla de nuestro teclado.
\end{sphinxadmonition}

\sphinxAtStartPar
Seleccionar \sphinxstyleemphasis{UTF\sphinxhyphen{}8} y luego \sphinxstyleemphasis{\#Latino1 y Latino5 \sphinxhyphen{} Europa Occidental y lenguas turcas} y ahora el tipo que desees. Yo elijo \sphinxstyleemphasis{VGA} como he comentado anteriormente y un tamaño de \sphinxstyleemphasis{8x16}. Verás que en este momento la consola presenta los tipos \sphinxstyleemphasis{VGA}.
\begin{enumerate}
\sphinxsetlistlabels{\arabic}{enumi}{enumii}{}{.}%
\setcounter{enumi}{11}
\item {} 
\sphinxAtStartPar
A continuación vamos a instalar el sistema básico. Si quieres instalar un sistema de escritorio lo más práctico es utilizar \sphinxstyleemphasis{tasksel}. Si estás instlando un sistema mínimo, sin entorno gráfico, en una \sphinxstyleemphasis{Raspberry Pi Zero}, por ejemplo, salta al punto 13.

\end{enumerate}

\begin{sphinxVerbatim}[commandchars=\\\{\},numbers=left,firstnumber=1,stepnumber=1]
\PYGZdl{}\PYG{+w}{ }sudo\PYG{+w}{ }tasksel
\end{sphinxVerbatim}

\sphinxAtStartPar
Y seleccionamos mediante la \sphinxstyleemphasis{barra de espacio} \sphinxstyleemphasis{Debian desktop environment} y el escritorio que más te guste, teniendo en cuenta que el escritorio que elijas puede consumir muchos recursos. Yo elijo \sphinxstyleemphasis{LXQT} porque es el \sphinxstyleemphasis{original} del que utiliza \sphinxstyleemphasis{Raspberry Pi OS} y sobre todo porque consume muy pocos recursos. Este paso dura bastante tiempo y es posible que la pantalla se desconfigure. No te preocupes y deja que siga hasta que termine.
\begin{enumerate}
\sphinxsetlistlabels{\arabic}{enumi}{enumii}{}{.}%
\setcounter{enumi}{12}
\item {} 
\sphinxAtStartPar
Cambiamos la hora a nuestra hora local. Para ello tecleamos \sphinxstyleemphasis{timedatectl list\sphinxhyphen{}timezones} y buscamos cual es nuestra zona. En mi caso \sphinxstyleemphasis{Europa/Madrid} y hacemos

\end{enumerate}

\begin{sphinxVerbatim}[commandchars=\\\{\},numbers=left,firstnumber=1,stepnumber=1]
\PYGZdl{}\PYG{+w}{ }sudo\PYG{+w}{ }timedatectl\PYG{+w}{ }set\PYGZhy{}timezone\PYG{+w}{ }Europe/Madrid
\PYGZdl{}\PYG{+w}{ }date
\end{sphinxVerbatim}

\sphinxAtStartPar
Y verás que la hora ya está actualizada a tu zona.
\begin{enumerate}
\sphinxsetlistlabels{\arabic}{enumi}{enumii}{}{.}%
\setcounter{enumi}{13}
\item {} 
\sphinxAtStartPar
Ya solo queda reiniciar el sistema para tener un Sistema Operativo instalado en nuestra \sphinxstyleemphasis{RPi}.

\end{enumerate}

\begin{sphinxVerbatim}[commandchars=\\\{\},numbers=left,firstnumber=1,stepnumber=1]
\PYGZdl{}\PYG{+w}{ }sudo\PYG{+w}{ }reboot
\end{sphinxVerbatim}
\begin{enumerate}
\sphinxsetlistlabels{\arabic}{enumi}{enumii}{}{.}%
\setcounter{enumi}{14}
\item {} 
\sphinxAtStartPar
Entramos en nuestro nuevo sistema con nuestro usuario y contraseña. Normalmente tardará unos minutos, mientras se configura, hasta que se vea el \sphinxstyleemphasis{prompt}, por último, abrimos un terminal, si estamos en un entorno gráfico, y actualizamos y limpiamos nuestro sistema.

\end{enumerate}

\begin{sphinxVerbatim}[commandchars=\\\{\},numbers=left,firstnumber=1,stepnumber=1]
\PYGZdl{}\PYG{+w}{ }sudo\PYG{+w}{ }apt\PYG{+w}{ }update\PYG{+w}{ }\PYG{o}{\PYGZam{}\PYGZam{}}\PYG{+w}{ }sudo\PYG{+w}{ }apt\PYG{+w}{ }upgrade\PYG{+w}{ }\PYG{o}{\PYGZam{}\PYGZam{}}\PYG{+w}{ }sudo\PYG{+w}{ }apt\PYG{+w}{ }full\PYGZhy{}upgrade\PYG{+w}{ }\PYG{o}{\PYGZam{}\PYGZam{}}\PYG{+w}{ }sudo\PYG{+w}{ }apt\PYG{+w}{ }clean\PYG{+w}{ }\PYG{o}{\PYGZam{}\PYGZam{}}\PYG{+w}{ }sudo\PYG{+w}{ }apt\PYG{+w}{ }autoremove
\end{sphinxVerbatim}

\sphinxAtStartPar
Ahora ya solo queda configurar el Sistema como más te guste y con las aplicaciones que necesites.



\renewcommand{\indexname}{Índice}
\printindex
\end{document}